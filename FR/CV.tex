\documentclass[10pt,a4paper]{CV}

\usepackage[francais]{babel} 
\usepackage[utf8]{inputenc}
%\usepackage[T1]{fontenc}

\Language{francais}

\hypersetup{
	pdfauthor   = {Sylvain LAFON},
	pdftitle    = {Curriculum Vitae},
	pdfcreator  = {TeXMaker},
	pdfproducer = {TeXMaker}
}


\definecolor{urlcolor}{rgb}{0, 0, 0.5}
\Detailed{0}

\geometry{
	hmargin=2.3cm,
	vmargin=2.5cm
}

\setlength{\colwidth}{2.5cm}

\smallskipamount=2.2mm	
\medskipamount=4.5mm
\bigskipamount=10mm

\begin{document}

\begin{heading}
	\Name{Sylvain LAFON}
	\Address{640, avenue de Bredasque\\
		13090 AIX-EN-PROVENCE\\}
	\Telephone{(+33)6.68.91.96.56}
	\Email{sylvain.lafon@free.fr}		
	\DateOfBirth{29 Septembre 1992\\}
	\Gender{M}	

\end{heading}

\begin{section}{Langages et Outils}
	\singleEntry{$\cdot$ Programmation : C/C++, Unity3D, WPF/XAML, C\#, Java, Android, ObjC}
	\singleEntry{$\cdot$ Langages du web : XML/DTD/XSLT, HTML/CSS, PHP, Javascript, JSON}
	% \singleEntry{$\cdot$ Technologies du web : jQuery, Symfony, Bootstrap, Node.js}
	\singleEntry{$\cdot$ Systèmes : GNU/Linux (Debian-like), Windows, Apple}
	\singleEntry{$\cdot$ Documentation : UML, Merise, Dot/Graphviz, Doxygen, Sandcastle, MediaWiki}
	\singleEntry{$\cdot$ Bureautique : Office, LibreOffice, \LaTeX}
	\singleEntry{$\cdot$ Bases de données : Oracle, MySQL, Access} % , NoSQL (MongoDB)
	\singleEntry{$\cdot$ Gestionnaires de version : git, svn (dépôts maison et GitHub/Bitbucket)}
	\singleEntry{$\cdot$ Scripts : Applescript, Visual Basic, bash, batch}
\end{section}

\begin{section}{Expérience professionnelle}
	\begin{entry}
		\Date{2016 -- (2019)}
		\Duration{En cours}
		\Place{\href{https://www.manzavision.com/}{Manzavision}}
		\Locality{Aix-en-Provence}
		\Activity{Lead Game Programmer Unity3D XR}
		\Activity{Spécialisation dans la technologie XR (VR, AR, MR) sur Unity}
		\Activity{$\cdot$ Réalisation de plusieurs tools et moteurs dans Unity3D}
		\Activity{$\cdot$ Réalisation d'un backoffice (site web/base de données/API) et d'un launcher WPF}
		\Activity{$\cdot$ Veille technologique, R\&D, Documentation}
		\Activity{$\cdot$ Cibles : GearVR, Cardboard (iOS et Android), Daydream, MagicLeap, HoloLens, HTC Vive, Vive Focus, Oculus Go, Oculus Rift, Oculus Quest, PSVR, Kinect}
	\end{entry}
	\begin{entry}
		\Date{2012 -- 2016}		
		\Duration{4 ans}
		\Place{\href{http://www.manzalab.com/}{Manzalab}}
		\Locality{Paris}
		\Activity{Game Programmer Unity3D}
		\Activity{$\cdot$ Intégration, Tool (Unity3D et externe), Moteur, Gameplay, UI, référence pour git, support sur les autres projets, intégration de webapi et plugins tiers}
		\Activity{$\cdot$ Cibles : iOS, PC, smartphones et tablettes (Apple comme Android)}
	\end{entry}
	\begin{entry}
		\Date{2012}
		\Duration{3 mois}
		\Place{MyOsteopathe}
		\Locality{Paris}
		\Activity{Programmeur Web}
		\Activity{Création d'un site web capable de se modifier intégralement à l'aide d'un panneau d'administration en ligne}
	\end{entry}
	\begin{entry}
		\Date{2010 et 2011}
		\Duration{2 mois}
		\Place{Société Générale}
		\Locality{Paris}
		\Activity{Auxiliaire de vacances}
		\Course{Réception}
		\Course{Archivage et Numérisation}
	\end{entry}
\end{section}

\begin{section}{Formation}
	\begin{entry}
		\Date{2012 -- 2014}
		\Place{\href{https://www.isartdigital.com/fr/}{ISART Digital}, \href{http://www.rncp.cncp.gouv.fr/grand-public/visualisationFiche?format=fr\&fiche=28719}{Game Programming}}
		\Locality{Paris}
		\Activity{Entrée directe en 3ème année}
		\Activity{$\cdot$ Outils du jeu vidéo (Physique, IA, Maths, Algo, Réseau, Son)}
		\Activity{$\cdot$ Réalisation d'un jeu vidéo en équipe avec d'autres étudiants de l'école}
	\end{entry}
	\begin{entry}
		\Date{2012}
		\Place{DUT Informatique}
		\Locality{\href{http://www.iut-orsay.fr/}{IUT d'Orsay}} % :p
		\Activity{Programmation (C99, C++, Java, C99 avec unistd, ..)}		
		\Activity{Apprentissage de méthodes d'analyse}\Activity{Système/Réseaux/Architecture des ordinateurs}		
	\end{entry}	
	\begin{entry}
		\Date{2010}
		\Place{Baccalauréat S-SI spécialité Mathématiques}
		\Locality{Lycée de Montegeron}
		\Activity{Apprentissage de nombreux langages de programmation en autodidacte (C99, C++ et langages du web)}
	\end{entry}
\end{section}

\begin{section}{Langues et centres d'intérêt}
	\singleEntry{$\cdot$ Anglais : lu, écrit, compris, parlé}
	\singleEntry{$\cdot$ Escrime, Programmation, GameJams/Hackathons}
	\singleEntry{$\cdot$ Jeux de plateau, Jeux de rôles, Jeux vidéos, Mangas, Soirées bar, Apprendre (tout)}
\end{section}

\end{document}
