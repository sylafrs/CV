\documentclass[10pt,a4paper]{CV}

\usepackage[francais]{babel} 
\usepackage[utf8]{inputenc}
%\usepackage[T1]{fontenc}

\Language{francais}

\hypersetup{
	pdfauthor   = {Sylvain LAFON},
	pdftitle    = {Curriculum Vitae},
	pdfcreator  = {TeXMaker},
	pdfproducer = {TeXMaker}
}


\definecolor{urlcolor}{rgb}{0, 0, 0.5}
\Detailed{0}

\geometry{
	hmargin=2.3cm,
	vmargin=2.5cm
}

\setlength{\colwidth}{2.5cm}

\smallskipamount=2mm	
\medskipamount=6mm
\bigskipamount=12.8mm		

\begin{document}

\begin{heading}
	\Name{Sylvain LAFON}
	\Address{52, avenue du Belvédère\\
		91800 BRUNOY\\}
	\Telephone{(+33)6.68.91.96.56}
	\Email{sylvain.lafon@free.fr}		
	\DateOfBirth{29 Septembre 1992\\}
	\Gender{M}	

\end{heading}

\begin{section}{Langages et Outils}
	\singleEntry{$\cdot$ Programmation : C/C++, Unity/C\#, Java, Android, ObjC}
	\singleEntry{$\cdot$ VR : GearVR, Vive, Oculus, Daydream, Cardboard, Playstation VR}
	\singleEntry{$\cdot$ Langages du web : XML/DTD/XSLT, HTML/CSS, PHP, Javascript}
	\singleEntry{$\cdot$ Technologies du web : jQuery, Symfony, Bootstrap, Node.js}
	\singleEntry{$\cdot$ Systèmes : GNU/Linux (Debian-like), Windows, Apple}
	\singleEntry{$\cdot$ Analyse : UML, Merise}
	\singleEntry{$\cdot$ Bureautique : Office, LibreOffice, \LaTeX}
	\singleEntry{$\cdot$ Bases de données : Oracle, MySQL, Access, NoSQL (MongoDB)}
	\singleEntry{$\cdot$ Gestionnaires de version : git, svn}
	\singleEntry{$\cdot$ Scripts : Applescript, Visual Basic, bash, batch}
\end{section}

\begin{section}{Expérience professionnelle}
	\begin{entry}
		\Date{2016}
		\Duration{En cours}
		\Place{\href{http://www.manzalab.com/}{Manzavision}}
		\Locality{Aix-en-Provence}
		\Activity{Lead Game Programmer}
		\Activity{Spécialisation dans la technologie VR sur Unity}
	\end{entry}
	\begin{entry}
		\Date{2012 -- 2016}		
		\Duration{4 ans}
		\Place{\href{http://www.manzalab.com/}{Manzalab}}
		\Locality{Paris}
		\Activity{Game Programmer}
		\Activity{Création et intégration de scripts pour différents projets sous Unity (iOS, PC, smartphones et tablettes (Apple comme Android))}
	\end{entry}
	\begin{entry}
		\Date{2012}
		\Duration{3 mois}
		\Place{\href{http://myosteopathe.com/}{MyOsteopathe}}
		\Locality{Paris}
		\Activity{Programmeur Web}
		\Activity{Création d'un site web capable de se modifier intégralement à l'aide d'un panneau d'administration en ligne}
	\end{entry}
	\begin{entry}
		\Date{2010 et 2011}
		\Duration{2 mois}
		\Place{Société Générale}
		\Locality{Paris}
		\Activity{Auxiliaire de vacances}
		\Course{Réception}
		\Course{Archivage et Numérisation}
	\end{entry}
\end{section}

\begin{section}{Formation}
	\begin{entry}
		\Date{2012 -- 2014}
		\Place{ISART Digital}
		\Country{Game Programming, Paris} % Oui, je sais ^^'
		\Activity{Entrée directe en 3ème année}
		\Activity{Outils du jeu vidéo (Physique, IA, Maths, Algo, Réseau, Son)}
		\Activity{Réalisation d'un jeu vidéo en équipe avec d'autres étudiants de l'école}
	\end{entry}
	\begin{entry}
		\Date{2012}
		\Place{DUT Informatique}
		\Country{\href{http://www.iut-orsay.fr/}{IUT d'Orsay}} % :p
		\Activity{Programmation avec différents langages dont tous ceux que je connaissais déjà}		
		\Activity{Apprentissage de méthodes d'analyse}\Activity{Système/Réseaux/Architecture des ordinateurs}		
	\end{entry}	
	\begin{entry}
		\Date{2010}
		\Place{Baccalauréat S-SI spécialité Mathématiques}
		\Activity{Apprentissage de nombreux langages de programmation en autodidacte (C/C++ et les langages du web)}
	\end{entry}
\end{section}

\begin{section}{Langues}
	\singleEntry{$\cdot$ Anglais : lu, écrit, compris}
\end{section}

\end{document}
